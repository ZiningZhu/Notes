\section{Wednesday 0606}

\subsection{Morphology: why do we need it? } (Graham Neubig)

\subsubsection{Neural factor-graph models for cross-lingual morphological tagging}
\begin{itemize}
	\item What is morphology? To tag the PoS, definite, pronoun type, gender, number, etc.
	\item Low-resource tagging: hard to use neural models.
	
	Cross-lingual transfer etc. can help.
	
	\item Baseline model: tag-set prediction (Cotterell+2017). Char-level bi-LSTM + word-level bidirectional LSTM on top.
	
	\item Problem: no way to capture the \TODO{?}
	
	\item Proposed model: facorial CRF + variable potentials from BiLSTM. The BiLSTM predicts label scores for all tags. 
	
	\item Advantages: (1) Generate arbitrary sequence of graphs; (2) \TODO{?}
	
	\item Model Factorization: neural, pairwise, transition
	
	\item Inference: loopy belief propagation. (dynamic programming)
	
	\item To predict tags: 
	
	\item Experiments: CRF transition / pairwise parameters should be initialized to 0 for stability.
\end{itemize}

\paragraph{Desiderata for morphology}
\begin{itemize}
	\item Explicitly capture orphological variation. e.g: different forms of the same word; consistency within phrases; elegantly predict probabilities for unknown words. 
\end{itemize}